\documentclass[12pt]{article}
\usepackage[utf8]{inputenc}
\usepackage{float}
\usepackage{amsmath}


\usepackage[hmargin=3cm,vmargin=6.0cm]{geometry}
%\topmargin=0cm
\topmargin=-2cm
\addtolength{\textheight}{6.5cm}
\addtolength{\textwidth}{2.0cm}
%\setlength{\leftmargin}{-5cm}
\setlength{\oddsidemargin}{0.0cm}
\setlength{\evensidemargin}{0.0cm}

%misc libraries goes here
\usepackage{tikz}
\usetikzlibrary{automata,positioning}

\begin{document}

\section*{Student Information } 
Full Name : Kadir CETINKAYA \\
Id Number : 2036457 \\

\section*{Graph}
\includegraphics[height=360pt]{graph.png}
We can see from the graph that here is a linear correspondence
between the emulated delay and the end-to-end delay. But the
slope of the graph is not one, since the emulated delay does not
consider the queueing delay, propagation delay, transmission delay
and the routing delay. Therefore the slope of the graph is greater
than one but since those delays are not the dominating factors when
compared with the network delays we introduced with tc, therefore
the graph is still linear.

\section*{Routing Tables}
\begin{tabular}{|cc|}
\hline
\multicolumn{2}{|c|}{\textbf{Routing Table of A}}\\
\hline
\textbf{Target} & \textbf{Next Hop}\\
A & A \\
B & B \\
C & B \\
D & B \\
E & B \\
\hline
\end{tabular}
\begin{tabular}{|cc|}
\hline
\multicolumn{2}{|c|}{\textbf{Routing Table of B}}\\
\hline
\textbf{Target} & \textbf{Next Hop}\\
A & A \\
B & B \\
C & C \\
D & C \\
E & C \\
\hline
\end{tabular}
\begin{tabular}{|cc|}
\hline
\multicolumn{2}{|c|}{\textbf{Routing Table of C}}\\
\hline
\textbf{Target} & \textbf{Next Hop}\\
A & B \\
B & B \\
C & C \\
D & D \\
E & D \\
\hline
\end{tabular}
\\
\begin{tabular}{|cc|}
\hline
\multicolumn{2}{|c|}{\textbf{Routing Table of D}}\\
\hline
\textbf{Target} & \textbf{Next Hop}\\
A & C \\
B & C \\
C & C \\
D & D \\
E & E \\
\hline
\end{tabular}
\begin{tabular}{|cc|}
\hline
\multicolumn{2}{|c|}{\textbf{Routing Table of E}}\\
\hline
\textbf{Target} & \textbf{Next Hop}\\
A & D \\
B & D \\
C & D \\
D & D \\
E & E \\
\hline
\end{tabular}

\section*{Desing}
Before writing the code, we designed our protocol and decided to
put a total of 10 bytes header per packet. In the first byte we
stored the packet's source/destination information and whether
the packet is a acknowledgement or a data packet. To do that
we used radix 5 representation, since our source/destination
info is a discrete value between A,B,C,D,E we mapped them to
0,1,2,3,4 and for ACK/Data packet difference we used only one bit,
1 for ACK and 0 for Data and calculated the byte using the formula
source+5*destinatin+25*isAck, since this value is always less than
256 we safely stored it in one byte. For the next 8 bytes we put
the timestamp of the sender when the packet had been generated.
And for the last byte we put the length of the data, which is
always less then 256 for our use in this homework and of course
we appended the payload to the header lastly.

After those arrengements we needed to map the 0,1,2,3,4 convention to
IP's of the machines, since routers had 2 interfaces and a machine could
only see one of them we decided to store the mapping as a (target,this,ip)
triplet in which we get the 'ip' of the 'target' which is reachable from 'this'
machine.

For the routing tables, since we know how to get the IP of a target from
0,1,2,3,4 mapping, we decided to store them also in the (target,this,nexthop)
triplet, in which we can get the 'nexthop', which is in $\{0,1,2,3,4\}$, to reach
'target' from 'this' node, the 'target' and 'this' are also in $\{0,1,2,3,4\}$.

To measure the end-to-end delay we are putting a timestamp for the package
creation time and measuring the time again as soon as we receive the packet
and look at the time difference. For this method to be sane we need to make
sure that sender and receiver have a synchronised time, to do that we used
the NTP(Network Time Protocol), we setup our server on node E and added A
as a peer to it and synced the clock of A to the E's. After NTP synch we 
checked their clock difference with clockdiff and achieved a 0/0ms clockdiff.
So after that our method would work without introducing any unexpected errors
to the measurement.

\section*{Experiment}
To get the end-to-end delay, we sent 100 packets from client to server
measured the time difference and get the average of those hundred packets
our means were, $42.0280623436$, $80.9206318855$, $162.675359249$, $321.32212162$
all in milliseconds for $10$,$20$,$40$,$80$ ms network delays respectively. After
that we calculated confidence intervals for each of our results, to do so
we calculated the standard deviation in each experiment multiplied it by $1.96$
and dividing it by 10, which is the square root of 100, length of our dataset.
So our confidence intervals were, $\pm 1.84840390578$ms, $\pm 3.1282490074$ms, 
$\pm 2.11410422494$ms, $\pm 2.07671289966$ms respectively, for each experiment.
Then plotted everypoint with those errors set as errorbars around the points.

During the homework, we designed and implemented all the parts together.

\end{document}
